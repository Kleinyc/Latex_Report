\documentclass[a4paper]{article}

\usepackage[english]{babel}
\usepackage[utf8]{inputenc}
\usepackage{fullpage}
\usepackage{amsmath}
\usepackage{graphicx}
\usepackage[colorinlistoftodos]{todonotes}
\usepackage{hyperref}
\usepackage{amssymb}
\usepackage{outline} \usepackage{pmgraph} \usepackage[normalem]{ulem}
\usepackage{graphicx} \usepackage{verbatim}
% \usepackage{minted} % need `-shell-escape' argument for local compile
\usepackage{ctex}
\usepackage{listings}
\title{
    \vspace*{1in}
    \includegraphics[width=2.75in]{figures/zhenglab-logo} \\
    \vspace*{1.2in}
    \textbf{\huge Weekly Work Report}
    \vspace{0.2in}
}

\author{Name \\
    \vspace*{0.5in} \\
    \textbf{NER \& RE} \\
    \vspace*{1in}
}

\date{\today}


\begin{document}

\maketitle
\setcounter{page}{0}
\thispagestyle{empty}
\newpage


\section{工作进展}

1、瑞金瑞金医院基于命名实体识别的关系抽取代码,两份代码都跑通了,其中涉及的都是一些细节的问题。

2、命名实体识别,没有正确的使用所里的数据

\section{问题}

数据部分的格式问题,因为其中主要是集成的包,暂时没有理解数据的输入格式问题


\section{RE}

模型输入包括两个实体的句子,输出句子中两个实体的类别,即对句子进行多分类。其中两个实体的关系类别一共有11类,包括赛题中的10类实体关系(正样本)和None关系(负样本)。

队伍二的代码比较简单,需要注意的问题是:个别语句因为版本变动,一些参数名称发生了变化。

队伍一的代码整体比较复杂,训练时间也更长一些,其他的语句问题也和队伍二一样,发生了一些变化。


%\section{Progress in this week}
%
%List what you have done in this week in detail.
%
%For example, maybe you performed some experiments this week. The following are the steps you took:
%
%\begin{description}
%\item [Step 1]
%Got up to welcome a new day.
%\item[Step 2]
%Opened your computer to start a new day's work.
%\item[Step 3]
%Got stuck with a very hard problem, like $e^{i \pi} + 1 = 0$.
%\item[Step 4]
%You searched online and realized some useful information like Figure~\ref{fig:google} shows. You asked other people for help and got the things done luckily.
%\end{description}



% If you don't cite any references, please comment the following two lines
%\bibliographystyle{ieee}
%\bibliography{ref.bib}

\newpage
\appendix
\section{队伍2}
\begin{lstlisting}[language=python]
	# 利用候选 relation 所在的句子训练字符级别的字向量
	word2idx = {'<pad>': 0, '<unk>': 1}
	word2idx, index_to_key, w2v_embeddings = train_word_embeddings(
	entity_pairs=chain(train_entity_pairs, test_entity_pairs),
	word2idx=word2idx,
	vector_size=100,
	epochs=1
	)

\end{lstlisting}


\end{document}